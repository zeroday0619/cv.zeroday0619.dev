% English CV base template (single-file)
% Compile: latexmk -pdf cv.tex
\documentclass[11pt,a4paper]{article}

\usepackage[a4paper,margin=1in]{geometry}
\usepackage[T1]{fontenc}
\usepackage[utf8]{inputenc}
\usepackage{lmodern}
\usepackage{microtype}
\usepackage{hyperref}
\usepackage{enumitem}
\usepackage{titlesec}
\usepackage{xcolor}

\hypersetup{
  colorlinks=true,
  urlcolor=blue,
  linkcolor=black,
  pdfauthor={Euiseo Cha},
  pdftitle={Euiseo Cha -- CV}
}

% Section formatting
\titleformat{\section}{\large\bfseries}{}{0em}{}[\titlerule]
\titlespacing*{\section}{0pt}{12pt}{8pt}

% Compact lists
\setlist[itemize]{noitemsep,topsep=2pt,leftmargin=*}

% --- Simple CV macros ---
\newcommand{\cvname}[1]{\begin{center}\Huge\bfseries #1\end{center}}
\newcommand{\cvcontact}[1]{\begin{center}\small #1\end{center}}

% Entry with title (role), org, location, dates, and bullets
\newcommand{\cventry}[5]{%
  \textbf{#1} \hfill {\small #4}\\
  \textit{#2} \hfill \textit{\small #3}\\
  \vspace{-4pt}
  \begin{itemize}[topsep=2pt,itemsep=2pt]
    #5
  \end{itemize}
  \vspace{2pt}
}

% Two-column skill row: Category : items
\newcommand{\cvskill}[2]{\textbf{#1}: #2\\}

\begin{document}

% ===== Header =====
\cvname{Euiseo Cha}
\cvcontact{
  Cheonan-si, South Korea \;|\;
  \href{mailto:escha@zeroday0619.dev}{escha@zeroday0619.dev} \;|\;
  +82\,010\,9061\,4607 \;|\;
  \href{https://github.com/zeroday0619}{github.com/zeroday0619} \;|\;
  \href{https://www.linkedin.com/in/euiseo-cha}{linkedin.com/in/euiseo-cha}
}

% ===== Summary =====
\section*{Summary}
Open-source enthusiast and Ubuntu Member with active contributions to the Ubuntu Community. As a forum moderator for Ubuntu Korea LoCo, I support community collaboration and knowledge sharing. I contribute code to open-source projects through GitHub and other collaboration platforms, and I'm committed to long-term growth in computer science and accelerated computing.

% ===== Experience =====
% \section*{Experience}
% \cventry{Job Title}{Company Name}{City, Country}{YYYY.MM -- Present}{
%   \item Impact-focused bullet: what you did + measurable result (e.g., reduced p95 latency by 35\%).
%   \item Scope/ownership bullet: systems, scale, responsibilities.
%   \item Tech stack bullet: Go, Java, PostgreSQL, Kubernetes, AWS, etc.
% }

% \cventry{Job Title}{Company Name}{City, Country}{YYYY.MM -- YYYY.MM}{
%   \item Achievement bullet with metric.
%   \item Another achievement.
% }

% ===== Projects (optional but recommended) =====
\section*{Projects}
\cventry{Discord TTS Bot}{koe-next (Team Project)}{GitHub}{Ongoing}{
  \item Python-based Discord text-to-speech bot with strict type hints and comprehensive type coverage.
  \item Dependency management via \texttt{uv}; code formatting and linting with \texttt{ruff}.
  \item Implemented test-driven development (TDD) approach for robust, bug-free design.
}
\cventry{Discord Markdown Parser}{discordown (Team Project)}{GitHub}{Ongoing}{
  \item Parses Discord markdown syntax and converts it to Abstract Syntax Tree (AST) representation.
  \item Follows same engineering principles: strict type hints, \texttt{uv} dependency management, \texttt{ruff} formatting, and TDD.
}
\cventry{LUKS Encryption CLI}{luksctl}{GitHub}{Ongoing}{
  \item Rust-based CLI tool for easily mounting and unmounting LUKS encrypted volumes.
  \item Features UUID-based mapper name auto-generation, auto-create mount points, and multi-language support.
  \item Published on crates.io; available on \href{https://crates.io/crates/luksctl}{crates.io/crates/luksctl}.
}
\cventry{Open Source Contribution}{uvloop (MagicStack)}{GitHub}{2022.06 -- 2022.09}{
  \item Contributed fixes to improve compatibility with Python 3.11+.
  \item Updated CI/testing to cover Python 3.11 development versions.
  \item Merged upstream: \href{https://github.com/MagicStack/uvloop/pull/473}{MagicStack/uvloop\#473}
}

% ===== Education =====
\section*{Education}
\textbf{B.S. (Candidate), Information and Communication Engineering} \hfill \small 2022.03 -- Present\\
\textit{Wonkwang University, South Korea} \hfill \textit{\small On leave of absence}

% ===== Skills =====
\section*{Skills}
\begin{itemize}[topsep=0pt,itemsep=1pt]
  \item \textbf{Languages}: Python, C/C++, Go, Rust
  \item \textbf{Backend}: FastAPI
  \item \textbf{Infrastructure}: Linux (Ubuntu, RHEL), Docker, LXD/LXC
  \item \textbf{Practices}: CI/CD, Security
\end{itemize}

% ===== Certifications (optional) =====
% \section*{Certifications}
% \begin{itemize}
%   \item Certification Name -- Issuer (YYYY)
% \end{itemize}

% ===== Publications / Talks (optional) =====
\section*{Publications \& Talks}
\begin{itemize}
  \item Building Canonical Landscape Server in LXD Container on Ubuntu 24.04 LTS -- UbuCon Korea 2024 (2024) \href{https://events.canonical.com/event/48/contributions/441/}{talk} | \href{https://youtu.be/dd27ED2AW60}{recording}
  \item Upgrading Legacy Ubuntu Servers: A Research Lab Migration Experience from 18.04 LTS to 24.04.2 LTS -- UbuCon Korea 2025 (2025) \href{https://events.canonical.com/event/126/contributions/698/}{talk}
\end{itemize}

% ===== Extra-curricular Activities =====
\section*{Extra-curricular Activities}
\begin{itemize}
  \item Ubuntu Member
  \item Ubuntu Korea Community -- Community Moderation Team (Forum Moderator)
  \item UbuCon Korea 2025 Organizer
  \item MiniDebConf Busan 2025 Team
\end{itemize}

% ===== Awards (optional) =====
% \section*{Awards}
% \begin{itemize}
%   \item Award Name -- Organization (YYYY)
% \end{itemize}

\end{document}
